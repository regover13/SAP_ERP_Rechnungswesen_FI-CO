\abk{GuV}{Gewinn- und Verlustrechnung} 
\abk{AO}{Abgabenordnung} 
\abk{HGB}{Handelsgesetzbuch}
\section{Grundlagen Rechnungswesen}

\subsection{Externes Rechnungswesen}
\label{ssec:externesRechnungswesen}
\abk{Financial Accounting}{externes Rechnungswesen, Finanzwesen -- FI}
Rechnungswesen dient allgemein der Dokumentation von Leistungsaustauschbeziehungen eines Unternehmens in deren wirtschaftlichen Umfeld. Produzierte Waren oder Dienstleistungen und finanzielle Mittel stehen im Austausch und werden mittels betrieblichen Rechnungswesens vollständig erfasst, um eine unternehmerische Handlungsfähigkeit wie Planung und Steuerung zu ermöglichen. Das externe Rechnungswesen ist Bestandteil des betrieblichen Rechnungswesens und wird insbesondere zur Rechenschaftslegung und Information gegenüber externen Berichtsempfängern verwendet, um die Sicht auf die Wirtschaftlichkeit eines Unternehmens abzubilden\footnote{Vgl. \cite{Wohe2000}, S. 717}. Externe Adressaten sind beispielsweise Gläubiger und Anteilseigner und Finanzbehörden\footnote{Vgl. \cite{Wohe2000}, S. 694}.

Für Finanzbehörden sind beispielsweise steuerrechtliche Aspekte wie z.B. Umsatz- und Gewinnsteuer relevant. Alle Stakeholder erwarten die Aussagefähigkeit einer Unternehmung hinsichtlich der Höhe des Vermögens, der Schulden, Gewinn und Verluste. 

Die Rechtsgrundlage des externen Rechnungswesens ist im Handelsrecht im  HGB (Handelsgesetzbuch) verankert\footnote{Vgl. \cite{Woltje2008}, S. 14}. Für alle kapitalmarktorientierten Konzerne und Gesellschaften, die Schuldtitel an die Börse gegeben haben bzw. an einer US-amerikanischen Börse gelistet sind, gelten ausserdem die international anerkannten Rechnungslegungsstandards (IFRS)\footnote{Vgl. \cite{Klein2010}, S. 14}. Nicht-kapitalmarktorientierte Konzerngesellschaften haben nach § 315a Abs. 3 HGB ein Wahlrecht zur Anwendung von IFRS\footnote{Vgl. \cite{Funk2008}, S. 34}\footnote{Vgl. \cite{Oehler2005}, S. 13 f.}.
Für Unternehmen besteht eine handelsrechtliche Buchführungspflicht um Informationsanforderungen von Behörden erfüllen zu können, ebenso wird durch die Buchführungspflicht die nötige Transparenz für kaufmännisches Handeln im Unternehmen selbst geschaffen.

Wesentlicher Bestandteil der Buchführung ist die Finanzbuchhaltung aus dieser der Jahresabschluss (Unternehmensbilanz,  Gewinn und Verlustrechnung) erstellt wird. Externes Rechnungswesen wird auch mit der Finanzbuchhaltung und Jahresabschluss gleichgesetzt\footnote{Vgl. \cite{Wohe2000}, S. 693}. Die Bilanz informiert über den Stand von Vermögen und Schulden eines Unternehmens. Die Finanzbuchhaltung dokumentiert
und erfasst alle Geschäftsfälle (Geld- und Güterströme) eines Geschäftsjahres.
Im Anhang 1 (Kapitel \ref{sec:Anhang1}, S. \pageref{sec:Anhang1}) werden die wichtigsten Merkmale des Rechnungswesens zusammengefasst und dem internes Rechnungswesen gegenübergestellt.

%\subsubsection{Grundsätze ordnungsmäßiger Buchführung und Bilanzierung}
%Die die GoB (Grundsätze ordnungsgemäßer Buchführung und Bilanzierung) beschreiben Regeln wie die Umsetzung der Buchführung im Unternehmen durchgeführt werden soll. Zu dem basieren diese Grundsätze auf ein in der Praxis angewandtes kaufmännisches Verhalten um Korrektheit der Informationen zu gewährleisten. Im HBG und AO (Abgabenordnung) finden sich wesentliche Grundsätze zur ordnungsgemäßer Buchführung und Bilanzierung wieder, allerdings sind diese Grundsätze nicht allumfassend in der Gesetzgebung (geschrieben) verankert, sondern basieren zum Teil auch auf ungeschriebene kaufmännischen Gepflogenheiten\footnote{Vgl. \cite{Wohe2000}, S. 733 ff}. Die GoB besagen, dass die Buchführung eines Unternehmens den Grundsätzen des Handelsrechts Folge leisten muss und das eine Buchführung nach dem formal und inhaltlich in Ordnung sein muss. 
%Definition nach §238 Abs. 1 HGB besagt, dass die Buchführung \begin{quote}[...] einem sachverständigen Dritten innerhalb angemessener Zeit einen Überblick über Geschäftsvorfälle und über die Lage des Unternehmens vermitteln kann.\end{quote}
%
%Nachfolgend werden die wesentlichen Grundsätze nach formellen und materiellen Grundsätzen dargestellt\footnote{Vgl. \cite{Wohe2000}, S. 734 ff}.
%\\
%\textbf{Formelle Grundsatz: Klarheit und Übersichtlichkeit}
%\begin{compactitem} 
%\item  nach geplanten Kontenplan
%\item  in einer lebenden Sprache (keine Abkürzungen)
%\item  nach Belegprinzip (jede Buchung braucht einen Beleg)
%\item  Offenlegung nachträglicher Veränderungen
%\item  Grundsatz der Einzelerfassung und Nachprüfbarkeit
%\item  zeitnahe Verbuchung der Positionen
%\item  Einhaltung gesetzlicher Aufbewahrungspflichten
%\end{compactitem}
%\textbf{Materielle Grundsatz: Vollständigkeit und Richtigkeit}
%\begin{compactitem}
%\item  lückenlose Erfassung aller Geschäftsvorfälle
%\item  Verbuchung aller Geschäftsvorfälle
%\item  keine gefälschte Buchungen erzeugen
%\item  Verbuchung alle Geschäftsvorfälle auf den jeweils zutreffenden Konten
%\end{compactitem}
%Neben den Grundsätzen der Buchführung existieren weitere Grundsätze für den Bereich der Bilanzierung. Diese haben zum Ziel, eine Bilanzwahrheit einzuhalten, welche nach den Prinzipien der Richtigkeit und Willkürfreiheit zu erfüllen ist\footnote{Vgl. \cite{Wohe2000}, S. 735 ff}.
%
% 
%%Hinsichtlich der Vollständigkeit der Grundsätze der Bilanzierung, ist noch der Ansatzgrundsatz und der Bewertungsgrundsatz zu erwähnen, welche eine weitere Detaillierungsgrad der Grundsätze ordnungsgemäßer Bilanzierung abbilden.
%
%\subsubsection{Gewinn- und Verlustrechnung}
%Die GuV (Gewinn- und Verlustrechnung) ist zusätzlich zu der Unternehmensbilanz ein Bestandteil des externen Rechnungswesen eines Unternehmens. Die GuV bezieht sich auf den finanziellen Erfolg und ist im wesentlichen eine Gegenüberstellung von Aufwendungen und Erträgen, die gemäß § 242 HBG am Ende eines Geschäftsjahres zu erbringen ist\footnote{Vgl. \cite{Schuler2006}}.
%
%Die GuV bemisst alle erfolgsrelevanten Informationen, die in einer Rechnungsperiode anfallen. Daher ist die periodengerechte Erfassung aller Geschäftsvorfälle (z.B. Inventur) von Bedeutung. Die Buchungen von Aufwänden und Erträgen in Erfolgskonten und deren Saldierung und Abschluss im GuV-Konto, fließen als Ergebnis über das Eigenkapital-Konto letztlich in die Schlussbilanzen der Unternehmen ein.
%\abk{GoB}{Grundsätze ordnungsgemäßer Buchführung und Bilanzierung}

%\subsubsection{Jahresabschluss nach HGB und IFRS}
%text

\subsection{Internes Rechnungswesen}
\label{ssec:internesRechnungswesen}
\abk{Management Accounting}{internes Rechnungswesen, Controlling -- CO}

Der Bereich des betrieblichen Rechnungswesens übernimmt eine besondere Dokumentations- und Kontrollfunktion in einer Wirtschaftsunternehmung.
Die externen orientierten Aufgaben sind meist durch gesetzliche Vorgaben eingeschränkt und dienen der Rechenschaftslegung gegenüber Gesellschaftern, Gläubigern, der Öffentlichkeit und dem Staat (Vgl. s. Kapitel \ref{ssec:externesRechnungswesen} auf S. \pageref{ssec:externesRechnungswesen}).
Die intern orientierten Aufgaben unterliegen hingegen keinerlei sachfremden und gesetzlichen Vorgaben und können so den individuellen Ansprüchen des Managements angepasst werden. Das interne Rechnungswesen (Betriebsbuchhaltung oder Management Accounting) dient der mengen- und wertmäßig Erfassung sowie\\ Überwachung aller in einem Unternehmen auftretenden Geld- und Leistungsströme. Es ist typischerweise kurzfristig ausgerichtet und weist kürzere Abrechnungszeiträume als das Geschäftsjahr des externen Rechnungswesens auf\footnote{Vgl. \cite{Wohe2000}, S. 853}\footnote{Vgl. \cite{Schierenbeck2008}, S. 799}. Es kommt zu Überschneidungen 
und Berührungspunkten zwischen externen und internen Adressaten\footnote{Vgl. \cite{Muller2006}, S. 17}.
Im Anhang 1 (Kapitel \ref{sec:Anhang1}, S. \pageref{sec:Anhang1}) werden die wichtigsten Unterscheidungsmerkmale zwischen dem internen und dem externen Rechnungswesen dargestellt.\newpage
\noindent Es ergeben sich drei Hauptaufgaben im internen Rechnungswesen\footnote{Vgl. \cite{Schierenbeck2008}, S. 799}\footnote{Vgl. \cite{Baetge2008}, S. 72 f.}:
\begin{compactitem}
\item[1.] Die Ermittlung des kurzfristigen Betriebserfolgs
\item[2.] Kontrolle der Wirtschaftlichkeit und Budgetierung
\item[3.] Rechnerische Fundierung unternehmenspolitischer Entscheidungen
\end{compactitem}
%\subsubsection{Ermittlung des kurzfristigen Betriebserfolgs}
Zur Erfüllung der genannten Hauptaufgaben der internen Unternehmungsrechnung kommen die Systeme der Kosten- und Leistungsrechnung zum Einsatz\footnote{Vgl. \cite{Schierenbeck2008}, S.~800}. Nach Jörg Wöltje besteht das interne Rechnungswesen aus der Kostenrechnung, der Leistungsrechnung und der Erfolgsrechnung. Die Kostenrechnung dokumentiert die Verbrauchsseite, also die entstandenen Kosten des Produktionsprozesses. Die Leistungsrechnung dokumentiert die Entstehungsseite, also die erzielten Erlöse des Produktionsprozesses. Stellt man Verbrauchs- und Entstehungsseite gegenüber, spricht man von einer Erfolgsrechnung. Die kalkulatorische Erfolgsrechnung als Hauptbestandteil des internen Rechnungswesens wird für einzelne Produkte, Produktgruppen oder Unternehmensbereiche erstellt und soll für kurze Abrechnungszeiträume den Betriebserfolg als Saldo der bewerteten Periodenleistungen und der Periodenkosten ermitteln\footnote{Vgl. \cite{Woltje2008}, S.~179}.
Kernstück ist dabei die Kostenrechnung, die bei entsprechender Ausgestaltung den Prozess der Kostenentstehung schrittweise verfolgt und eine rechnerische Aufgliederung des Kostengefüges 
\begin{compactitem}
\item nach Kostenarten (Welche Kosten fallen an?), 
\item nach Kostenstellen (Wo fallen welche Kosten an?) und 
\item nach Kostenträgern (Wofür, d.h. für welche Leistungen fallen Kosten an?)
\end{compactitem} ermöglicht\footnote{Vgl. \cite{Schierenbeck2008}, S.~799}\footnote{Vgl. \cite{Ossadnik2008}, S~52 ff.}.
%
%
%\subsubsection{Kontrolle der Wirtschaftlichkeit und Budgetierung}
%Wie wirtschaftlich wird in den Kostenstellen gearbeitet?
%Aufgrund ihrer internen Ausrichtung fällt der kalkulatorischen Erfolgsrechnung die Aufgabe zu, den Ablauf der Unternehmungsprozesse zu überwachen. Klein sieht in der Kontrolle die Überwachung von Dispositionen untergeordneter Instanzen. Dabei kann Kontrolle definiert werden als die \glqq Durchführung eines Vergleichs zwischen einer zu prüfenden und einer Plangröße\grqq \footnote{Vgl. \cite{Klein1999}, S.~22}.
%Dieses Instrument zeigt, welche Produkte am stärksten am Umsatz beteiligt sind. Auch wird klar, welche Produktelemente die höchsten Kosten verursachen und deshalb besonders beachtet werden sollten. So werden Schwachstellen und Unwirtschaftlichkeiten im Leistungserstellungsprozesses erkannt\footnote{Vgl. \cite{Woltje2008}, S.~196 f.}. Mit dem Ziel, vorgegebene Kostendeckungs-, Gewinn- oder Rentabilitätsziele sicher zu stellen, wird die kalkulatorische Erfolgsrechnung auch für die Budgetierung von Kosten und Erlösen eingesetzt\footnote{Vgl. \cite{Schierenbeck2008}, S.~800}.
%
%\subsubsection{Rechnerische Fundierung unternehmenspolitischer Entscheidungen}
%Der internen Unternehmungsrechnung kommt hier die Aufgabe zu, das Zahlenmaterial für Entscheidungsrechnungen im Sinne von Planungsrechnungen zu liefern. Diese Funktion dient sowohl der Entscheidungsfindung als auch dem Entscheidungsvollzug\footnote{Vgl. \cite{Klein1999}, S.~22}\footnote{Vgl. \cite{Schierenbeck2008}, S.~800}.\\
%Zu den Hauptaufgaben zählen hier in erster Linie\footnote{Vgl. ebd., S.~800}: 
%\begin{compactitem}
%\item die Kalkulation von Preisen und Preisuntergrenzen, 
%\item die Planung des Mitteleinsatzes im Marketing, 
%\item die Produktions- und Absatzplanung, 
%\item die Produktionsdurchführungsplanung, 
%\item die Materialbereitstellungsplanung (einschließlich Wahl zwischen Eigenfertigung und Fremdbezug) \\sowie 
%\item die Investitions- und Finanzierungsprogrammplanung. 
%\end{compactitem} 



