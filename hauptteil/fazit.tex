\section{Fazit}
ERP-Software für das Finanzmanagement optimiert Abläufe des betrieblichen Rechnungswesen. Die SAP ERP-Anwendungen FI und CO können mit allen anderen Komponenten der Wertschöpfungskette integriert werden. Sie sparen Kosten und Zeit durch die Automatisierung von Geschäftsprozessen und schaffen im Rechnungswesen mehr Vertrauen durch eine zuverlässige Kontrolle finanzieller Risiken. 

Die SAP ERP Financials Module FI und CO sind elementare Bestandteile der Betriebswirtschaft und in jedem Unternehmen einsetzbar. Andere  Standard- oder eigenerstellte Softwarelösungen haben in deutschen Großunternehmen fast keine Bedeutung mehr\footnote{Vgl. \cite{Gleich2010}, S. 58}. Sie erfüllen sowohl die Vorgaben der gesetzlich vorgeschriebenen externen Rechungslegung, als auch die vielfältigen Anforderungen an interne Analysen und Reportings im interenen Rechungswesen. 

Das Konzernprogramm One.ERP der Deutschen Telekom AG setzt ein Prozess-Tool um, das flexibel, schnell, standardisiert und kosteneffizient ist. Dies hat den wesentlichen Vorteil, dass unabhängig vom Buchungskreis ein Geschäftsvorfall immer den gleichen Buchungsprozess bedingt. Für die Tochter Telekom Deutschland GmbH etwa bedeutet dies, dass die IT-Systeme von drei ehemals \glqq unabhängigen\grqq \ Gesellschaften (T-Home, T-Mobile, Geschäftskunden) zukünftig einheitlich abgebildet werden. Es muss nur noch ein System gepflegt werden, Rollouts werden nur noch einmal erforderlich sein und die IT-Kosten werden gesenkt. Bei zukünftigen Organisationsänderungen entfallen vielfältige Abstimmungsrunden, da es nur den einen Prozess gibt. Für alle Legaleinheiten des Konzerns, wird gerade durch die einheitliche Implementierung des Rechnungswesens in SAP ERP Financials, ein single point of truth geschaffen, welcher eine erstmals einheitliche Basis für die Geschäftssteuerung liefert.

Um langfristig im internationalen Wettbewerb bestehen zu können, sind im Zwang permanenter Veränderungen, agile und flexible Reaktionen unumgänglich. One.ERP ist für die Zukunft der Deutsche Telekom AG alternativlos.